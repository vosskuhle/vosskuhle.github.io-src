In turbulent suspensions, collision rates determine how rapidly particles coalesce or react with each other.
To determine the collision rate, many numerical studies rely on the ghost collision approximation (GCA), which simply records how often pairs of point particles come within a threshold distance.
In many applications, the suspended particles stick (or in the case of liquid droplets, coalesce) upon collision, and it is the frequency of first contact which is of interest.
If a pair of “ghost” particles undergoes multiple collisions, the GCA may overestimate the true collision rate.
Here, using fully resolved direct numerical simulations of turbulent flows at moderate Reynolds number ($Re_\lambda = 130$), we investigate the prevalence and properties of multiple collisions.
We find the probability $P(N_c)$ for a given pair of ghost particles to collide Nc times to be of the form $P(N_c) = \beta \alpha^{N_c}$for $N_c>1$, where $\alpha$ and $\beta$ are coefficients which depend upon the particle inertia.
We also investigate the statistics of the times that ghost particles remain in contact.
We show that the probability density function of the contact time is different for the first collision.
The difference is explained by the effect of caustics in the phase space of the suspended particles.
We demonstrate that, as a result of multiple collisions, the GCA leads to a small, but systematic overestimate of the collision rate, which is of the order of $\sim 15\%$ when the particle inertia is small, and slowly decreases when inertia increases.
