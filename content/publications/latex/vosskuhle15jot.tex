The use of simplified models of turbulent flows provides an appealing possibility to study the collision rate of turbulent suspensions, especially in conditions relevant to astrophysics, which require large timescale separations.
To check the validity of such approaches, we used a direct numerical simulation (DNS) velocity field, which satisfies the Navier--Stokes equations (although it neglects the effect of the suspended particles on the flow field), and a kinematic simulation (KS) velocity field, which is a random field designed so that its statistics are in accord with the Kolmogorov theory for fully-developed turbulence.
In the limit where the effects of particle inertia (characterised by the Stokes number) are negligible, the collision rates from the two approaches agree.
As the Stokes number ${\rm St}$ increases, however, we show that the DNS collision rate exceeds the KS collision rate by orders of magnitude.
We propose an explanation for this phenomenon and explore its consequences.
We discuss the collision rate $R$ for particles in high Reynolds number flows at large Stokes number, and present evidence that $R\propto \sqrt{{\rm St}}$.
