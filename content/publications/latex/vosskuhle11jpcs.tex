Most studies of collisions in turbulent flows are based on the \ldq ghost collision\rdq approximation, whereby one follows a number of particles, and simply counts the number of times the distance between two particles becomes less than the sum of their radii; particles are kept in the flow after they collided.
We discuss here the limitations of this approximation, and demonstrate, using a simple model flow, that it leads to overestimates of the real collision rate by as much as $\sim 30\,\%$ at small Stokes numbers.
