This thesis is devoted to the mechanisms leading to strong collision rates of inertial particles in turbulent suspensions.
Our work is based on simulating the motion of particles, using both direct numerical simulations of the Navier--Stokes equations, and a simpler model (kinematic simulations).
This subject is important for many applications, in industrial as well as natural (astrophysical, geophysical) contexts.
We revisit the ghost collision approximation (GCA), widely used to determine the rate of collisions in numerical simulations, which consists in counting how many times the centers of two particles come within a given distance.
Theoretical arguments suggested that this approximation leads to an overestimate of the real collision rate.
This work provides not only a quantitative description of this overestimate, but also a detailed understanding of the error made using the GCA.
We find that a given particle pair may undergo multiple collisions with a relatively high probability.
This is related to the observation that in turbulent flows, particle pairs may stay close for a very long time.
We have provided a full quantitative characterization of the time spent together by pairs of particles.
A second class of results obtained in this thesis concerns a quantitative understanding of the very strong collision rates often observed.
We demonstrate that when the particle inertia is not very small, the ``sling/caustics'' effect, i.e., the ejection of particles from energetic vortices in the flow, is responsible for the high collision rates.
The preferential concentration of particles in some regions of space plays in comparison a weaker role.
