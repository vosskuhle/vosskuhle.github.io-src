The work at hand investigates statistical properties of two-dimensional turbulence.
The exploration of two-dimensional flows is driven by two reasons.
On the one hand, flows in many geophysical systems and in plasma physics can be described as being quasi-two-dimensional, which explains the practical interest in this field of research.
On the other hand, two-dimensional turbulence shows some striking differences as compared to the three-dimensional case, like the emergence of large-scale coherent structures.
Despite these differences there are also universal features that the two have in common and one is eager to find them, as those universal features might be useful to solve the outstanding problem of turbulence.

In particular, this work deals with two rather distinct approaches.
First, an established technique for the investigation of the three-dimensional turbulent energy cascade [see Friedrich; Peinke, Phys.~Rev.~Lett.~\textbf{78}, 863 (1997)] is extended to the case of two-dimensional turbulence in the inverse energy cascade regime.
Evidence is given, that the velocity increments $v(r)$ may be described by a Markov process, given the separation length $r$ is greater than a certain length, the so called Markov--Einstein length.
Furthermore it is shown, that the velocity increments obey a Fokker--Planck equation.
This equation is deduced and inspires a simple approximation, which is able to reconstruct characteristic features of the increment statistics.

A second part is dedicated to the Lundgren--Monin--Novikov hierarchy (LMN hierarchy).
The LMN hierarchy is a system of partial differential equations describing the evolution of the $N$-point probability density functions of divers properties of fluid flows, e.g.~the velocity or the velocity increments.
It can be directly obtained from the Navier--Stokes equations.
This work studies the hierarchy for the vorticity and the vorticity increments in two-dimensional flows.
The appearing terms, that cannot be handled analytically, are investigated numerically.
This procedure leads to promising suggestions concerning a closure of the LMN hierarchy.
